\documentclass[11pt,a4paper,sans]{moderncv}        % possible options include font size ('10pt', '11pt' and '12pt'), paper size ('a4paper', 'letterpaper', 'a5paper', 'legalpaper', 'executivepaper' and 'landscape') and font family ('sans' and 'roman')
\usepackage{lmodern}
% moderncv themes
\moderncvstyle{banking}                            % style options are 'casual' (default), 'classic', 'oldstyle' and 'banking'
\moderncvcolor{blue}                              % color options 'blue' (default), 'orange', 'green', 'red', 'purple', 'grey' and 'black'
%\renewcommand{\familydefault}{\sfdefault}         % to set the default font; use '\sfdefault' for the default sans serif font, '\rmdefault' for the default roman one, or any tex font name
\nopagenumbers{}                                  % uncomment to suppress automatic page numbering for CVs longer than one page

% character encoding
\usepackage[utf8]{inputenc}
%\usepackage{hyperref}
%\usepackage{pdfpages}
% if you are not using xelatex ou lualatex, replace by the encoding you are using
%\usepackage{CJKutf8}                              % if you need to use CJK to typeset your resume in Chinese, Japanese or Korean
\usepackage{multicol}

\usepackage{xcolor}
% adjust the page margins
\usepackage[scale=0.9,top=1.5cm, bottom=0.5cm]{geometry}
% \usepackage[scale=0.75]{geometry}
%\setlength{\hintscolumnwidth}{3cm}                % if you want to change the width of the column with the dates
%\setlength{\makecvtitlenamewidth}{10cm}           % for the 'classic' style, if you want to force the width allocated to your name and avoid line breaks. be careful though, the length is normally calculated to avoid any overlap with your personal info; use this at your own typographical risks...
\usepackage{xpatch}
\xpatchcmd\cventry{,}{}{}{}

% personal data

\name{Ivan}{Ogloblin}                               % optional, remove / comment the line if not wanted
\firstname{Ivan} % Your first name
\lastname{Ogloblin} % Your last name

% All information in this block is optional, comment out any lines you don't need
\title{Curriculum Vitae}

% \address{70 Absolute Ave.}{L4Z 0A4 Mississauga}{Canada}% optional, remove / comment the line if not wanted; the "postcode city" and and "country" arguments can be omitted or provided empty
\vspace*{3mm}
% optional, remove / comment the line if not wanted
% \phone[fixed]{+2~(345)~678~901}                    % optional, remove / comment the line if not wanted
% \phone[fax]{+3~(456)~789~012}                      % optional, remove / comment the line if not wanted
%  \homepage{linkedin.com/in/jondoe}                         % optional, remove / comment the line if not wanted
%\social[linkedin]{AlyaNovikova}
% \extrainfo{additional information}                 % optional, remove / comment the line if not wanted
%photo[64pt][0.4pt]{picture}                       % optional, remove / comment the line if not wanted; '64pt' is the height the picture must be resized to, 0.4pt is the thickness of the frame around it (put it to 0pt for no frame) and 'picture' is the name of the picture file
% \quote{Some quote}                                 % optional, remove / comment the line if not wanted

% to show numerical labels in the bibliography (default is to show no labels); only useful if you make citations in your resume
%\makeatletter
%\renewcommand*{\bibliographyitemlabel}{\@biblabel{\arabic{enumiv}}}
%\makeatother
%\renewcommand*{\bibliographyitemlabel}{[\arabic{enumiv}]}% CONSIDER REPLACING THE ABOVE BY THIS

%----------------------------------------------------------------------------------
%           footer
%----------------------------------------------------------------------------------
% bibliography with mutiple entries
%\usepackage{multibib}
%\newcites{book,misc}{{Books},{Others}}
%----------------------------------------------------------------------------------
%            content
%----------------------------------------------------------------------------------
%\makecvfooter
\begin{document}
	%\begin{CJK*}{UTF8}{gbsn}                          % to typeset your resume in Chinese using CJK
	%-----       resume       ---------------------------------------------------------
	\vspace*{-15mm}
	\makecvtitle
	\vspace*{-15mm}
	
	\section{Education}
	\cventry{}{}{Bachelor of Science in Computer Science and Software Engineering}{Sept 2019 - July 2023}{\hspace*{-2.5 mm} Saint-Petersburg State University}{}
	\cventry{}{}{Master of Science in Mathematics}{Sept 2022 - April 2025}{\hspace*{-2.5 mm} Pontifical Catholic University of Rio de Janeiro}{}
	{}{}
	
	\section{Work Experience}
	
	\cventry{}{}{Quantum Software Engineer}{February 2024 - current\vspace{-1.0em}}{}{
		% Detailed achievements:%
		%\begin{itemize}
		I am working in a growing startup QC Design for over a year and managing up to 4 people's work integration into the software. I implemented a modest algorithm for dealing with qubit's possibility of being  \href{https://www.nature.com/articles/s41586-022-05434-1}{\textcolor{blue}{leaked}} from the computational system. I brought a new infrastructure that allowed implementing algorithms in c++ and calling them from Python using CPython. I achieved more than a 10000x speedup for simulations of challenging noise models in the circuit. I added support for \href{https://arxiv.org/pdf/2402.09333}{\textcolor{blue}{Bosonic system}}'s noise and \href{https://arxiv.org/abs/2105.04478Clifford}{\textcolor{blue}{Spanning technique}} that allows for the simulation to converge not to an approximation but to a precise value of the Logical Error Rate for any coherent noise. I also worked on the belief-propagation decoders as well as matching decoders, which were challenging to implement efficiently.
	}
	%\end{itemize}}
	\cventry{}{}{Huawei Assistant Engineer, Developer}{October 2021 - January 2022\vspace{-1.0em}}{}{
		% Detailed achievements:%
		%\begin{itemize}
		Worked on backend C\#/.netASP/EntityFramework/Autofac + frontend 3js/react/VR. Developed a system of package communication with no delay that alternates between http and signalR requests.
		%\end{itemize}
	}
	\cventry{}{}{Yandex Developer Intern}{July - Sept 2021\vspace{-1.0em}}{}{
		% Detailed achievements:%
		%\begin{itemize}
		Worked in two teams on backend C++/Python/SQL. Developed a support system for training scripts to work with stored variable logs. Wrote tests for components that were used to prepare data for a neural network that makes recommendations.
	}

	\section{Projects}
	
	\cventry{}{}{Simulation of photonic quantum computing}{2023\vspace{-1.0em}}{}{
		Developed a web service dedicated to simulation of linear and non-linear optics for quantum computational models using Python and Django. Used \href{https://strawberryfields.ai/}{\textcolor{blue}{Strawberry fields}} as an underlying engine. (\href{https://github.com/StudioShader/SF_Composer}{\textcolor{blue}{github}})
	}
	\cventry{}{}{Undergraduate Thesis}{2022-2023\vspace{-1.0em}}{}{
		I did research on optimal schemes of entangling transformations in linear quantum optics using a genetic algorithm with GPUs on Pytorch. New schemes were obtained for finding the maximum entangled state, as well as for implementing gates equivalent to CX in the KLM protocol.
		\href{https://github.com/StudioShader/galopy/blob/master/slides(eng).pdf}{\textcolor{blue}{Presentation}}.
	}
	
	\cventry{}{}{Study of the Effect of Noise on Efficient Quantum Search Algorithms}{2022\vspace{-1.0em}}{}{
		Implemented \href{https://doi.org/10.1007/s11128-021-03165-2}{\textcolor{blue}{improved Grover's search algorithm}} with Qiskit and tested it's performance with different noise models. \href{https://github.com/StudioShader/QPSA/blob/main/presentation.pdf}{\textcolor{blue}{Presentation}}.
	}
	\cventry{}{}{Quantum Algorithms for VRP and VRPTW Problems}{2021\vspace{-1.0em}}{}{
		Worked on the problem of finding routes for drilling machines for oil production in collaboration with GazpromNeft. Found a reduction of this problem to QUBO. Used Qiskit to solve it using VQE and QAOA.
	}
	\cventry{}{}{Teacher Assistant}{2023\vspace{-1.0em}}{}{
		Created homework and course notes on the course "Introduction to Quantum Computation", for prof. \href{https://scholar.google.com/citations?user=UnZl40AAAAAJ&hl=en}{\textcolor{blue}{Sergey Tikhomirov}}.
	}
	\cventry{}{}{Quantum Computing and Quantum Information via NMR}{2022\vspace{-1.0em}}{}{
		I operated an NMR device encoding and entangling two qubits at the School of Experimental Physics of CBPF earning a  \href{https://github.com/StudioShader/StudioShader/blob/main/Certificados-21.pdf}{\textcolor{blue}{certificate}}.
	}
	\cventry{}{}{Qiskit Global Summer School 2022 - Quantum Excellence}{2022\vspace{-1.0em}}{}{
		I excelled at Qiskit Global Summer School 2022 which was dedicated to quantum simulations, earning a \href{https://www.credly.com/badges/3304071b-2191-46fe-9de6-0b1cc019a06f/public_url}{\textcolor{blue}{badge on Credly}}.
	}
		
	\section{Programming Skills and Languages}
		\begin{itemize}
			\item C++, Python, CPython, C\#, C, Java, JavaScript, HTML, CSS, Kotlin, Haskell, Scala, SQL, Lean
			\item Pybind, ASPnet, EntityFramework, Microsoft SQL Express, React, three.js, postgreSQL, Django, Bootstrap
			\item Git, Linux, Unity3D, SVN, Blender(3d modeling), protobuff, Shiny, Docker
			\item Russian (Native), English (Fluent), Portuguese (Speaking)
		\end{itemize}
		
		\vspace*{\fill}
		\name{}{}
		\title{}
		\phone[mobile]{+49~015162630532}
		\address{Bismarckring, 64}{Ulm}{Germany}
		\email{ivanogloblin2022@gmail.com}
		\social[github]{ivanoglo}   
		\makecvtitle
		%\section{Languages}
		%\begin{itemize}
		%	\item Russian (Native), English (Upper-Intermediate)
		%\end{itemize}
		%\makecvfoot
	\end{document}